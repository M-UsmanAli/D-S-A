\documentclass[11pt]{article}            % Report class in 11 points
\parindent0pt  \parskip10pt             % make block paragraphs
\usepackage{graphicx}
\usepackage{listings}
\graphicspath{ {images/} }
\usepackage{graphicx} %  graphics header file
\begin{document}
\begin{titlepage}
    \centering
  \vfill
    \includegraphics[width=8cm]{uni_logo.png} \\ 
	\vskip2cm
    {\bfseries\Large
	Data Structure and Algorithms \\ (CS13217)\\
	
	\vskip2cm
	Lab Report 
	 
	\vskip2cm
	}    

\begin{center}
\begin{tabular}{ l l  } 

Name: & M.Usman Ali \\ 
Registration \#: & SEU-F16-135 \\ 
Lab Report \#: & 03 \\ 
 Dated:& 30-04-2018\\ 
Submitted To:& Sir. Usman Ahmed\\ 

 %\hline
\end{tabular}
\end{center}
    \vfill
    The University of Lahore, Islamabad Campus\\
Department of Computer Science \& Information Technology
\end{titlepage}


    
    {\bfseries\Large
\centering
	Experiment \# 3 \\

Stack with Array implementation \\
	
	}    
 \vskip1cm
 \textbf {Objective}\\ The objective of this session is to understand the various operations in stack using array structure in C++. 
 
 \textbf {Software Tool} \\
1. Language: C++\\

\section{Theory }              
Stacks are the most important in data structures. The notation of a stack in computer science is
the same as the notion of the Stack to which you are accustomed in everyday life. For example, a
recursion program on which function call itself, but what happen when a function which is
calling itself call another function. Such as a function ‘A’ call function ‘B’ as a recursion. So, the
firstly function ‘B’ is call in ‘A’ and then function ‘A’ is work. So, this is a Stack. This is a
Stack is First in Last Out data structure.

\section{Task}  
\subsection{Procedure: Task 3 }     
Write a C++ code to perform :
1. Insertion in stack
2. Deletion in stack
3. Display the stack

Create a complete menu for the above options and also create option for reusing it.

\begin{figure*}
\centering
  \includegraphics[width=12cm,height=6cm,keepaspectratio]{1255.png}
\caption{Link List}
\label{Figure:3}    
\end{figure*}

\begin{lstlisting}[language=C++]
#include<iostream>
using namespace std;
	int max=10;
	int a[10];
	int top=-1;
void push(int x)
	{
		if(top==10)
		{
			cout<<"Error:Over flow";
		}
		else
		{
			a[++top]=x;
		}
	}
void pop()
	{
		if(top==-1)
		{
			cout<<"Stack is empty";
		}
		else
		{
			top--;
		}
	}
	int print()
	{
		for(int i=0;i<top+1;i++)
		{
			cout<<a[i];
		}
	}
int main()
{	
	us: cout<<"1. push"<<endl;
	cout<<"2. pop"<<endl;
	cout<<"3. print"<<endl;
	cout<<"***************"<<endl;
	char k,y;
	while(y==0)
	{
	cin>>k;
	switch(k)
	{
		case '1':
		int a;
		cin>>a;
		push(a);
		goto us;
		break;
		case '2':
		pop();
		goto us;
		break;
		case '3':
		print();
		break;
}
}
}
\end{lstlisting}

\section{Conclusion}  
In this lab we learned how to create stack and display it on a screen and its various functions.

 
\end{document}                          % The required last line
