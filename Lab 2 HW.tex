\documentclass[11pt]{article}            % Report class in 11 points
\parindent0pt  \parskip10pt             % make block paragraphs
\usepackage{graphicx}
\usepackage{listings}
\graphicspath{ {images/} }
\usepackage{graphicx} %  graphics header file
\begin{document}
\begin{titlepage}
    \centering
  \vfill
    \includegraphics[width=8cm]{uni_logo.png} \\ 
	\vskip2cm
    {\bfseries\Large
	Data Structure and Algorithms \\ (CS13217)\\
	
	\vskip2cm
	Lab Report 
	 
	\vskip2cm
	}    

\begin{center}
\begin{tabular}{ l l  } 

Name: & M.Usman Ali \\ 
Registration \#: & SEU-F16-135 \\ 
Lab Report \#: & 02 \\ 
 Dated:& 30-04-2018\\ 
Submitted To:& Sir. Usman Ahmed\\ 

 %\hline
\end{tabular}
\end{center}
    \vfill
    The University of Lahore, Islamabad Campus\\
Department of Computer Science \& Information Technology
\end{titlepage}


    
    {\bfseries\Large
\centering
	Experiment \# 2 \\

Queue with Array implementation \\
	
	}    
 \vskip1cm
 \textbf {Objective}\\ The objective of this session is to understand the various operations on queues using array structure in C++. 
 
 \textbf {Software Tool} \\
1. Language: C++\\

\section{Theory }              


\section{Task}  
\subsection{Procedure: Task 2 }     
Write a C++ code to perform insertion and deletion in queue using arrays

Create a complete menu for the above options and also create option for reusing it.

\begin{figure*}
\centering
  \includegraphics[width=12cm,height=6cm,keepaspectratio]{125.png}
\caption{Link List}
\label{Figure:3}    
\end{figure*}

\begin{lstlisting}[language=C++]
#include<iostream>
#include<queue>
using namespace std;
int main()
{
	int x=-1,y=0;
	int arr[10];
	char op;
	cout<<"press 1 to push"<<endl;
	cout<<"press 2 to delete"<<endl;
	cout<<"press 3 to display"<<endl;
	cout<<"press 4 to exit"<<endl;
	line:
		cin>>op;
		switch(op){
			case'1':
				cout<<"enter no. to push"<<endl;
				x++;
				cin>>arr[x];
				cout<<"Pushed at "<<x<<endl;
				break;
			case'2':
				cout<<"deleting"<<endl;
				y++;
				break;
			case'3':
				for(int i=0;i<=x;i++)
				{
					cout<<arr[i]<<endl;
				}
				break;
			case'4':
				exit;
		}
		goto line;
}
\end{lstlisting}

\section{Conclusion}  
In this lab we learned how to create queue and display it on a screen and its various functions.

 
\end{document}                          % The required last line
