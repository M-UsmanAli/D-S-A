\documentclass[11pt]{article}            % Report class in 11 points
\parindent0pt  \parskip10pt             % make block paragraphs
\usepackage{graphicx}
\usepackage{listings}
\graphicspath{ {images/} }
\usepackage{graphicx} %  graphics header file
\begin{document}
\begin{titlepage}
    \centering
  \vfill
    \includegraphics[width=8cm]{uni_logo.png} \\ 
	\vskip2cm
    {\bfseries\Large
	Data Structures and Algorithms \\ ( CS09203 )\\
	
	\vskip2cm
	Lab Report 
	 
	\vskip2cm
	}    

\begin{center}
\begin{tabular}{ l l  } 

Name: & M.Usman Ali \\ 
Registration \#: & SEU-F16-135 \\ 
Lab Report \#: & 01 \\ 
 Dated:& 30-04-2018\\ 
Submitted To:& Mr. Usman Ahmed\\ 

 %\hline
\end{tabular}
\end{center}
    \vfill
    The University of Lahore, Islamabad Campus\\
Department of Computer Science \& Information Technology
\end{titlepage}


    
    {\bfseries\Large
\centering
	Experiment \# 1 \\

Introduction to Arrays and its operation\\
	
	}    
 \vskip1cm
 \textbf {Objective}\\ The objectives of this lab session are to understand the basic and various operations on arrays in C++. 
 
 \textbf {Software Tool}\\
1. Code Blocks with GCC compiler\\
2. \\
3. \\

\section{Theory }              We have already studied array in our computer programming course. We would be using the knowledge we learned there to implement different operation on arrays.
Traversing Linear Arrays:-
Let A be the collection of data elements stored in the memory of the computer. Suppose we want to print the contents of each element of A or suppose we want to count the number of elements of A with a given property. This can be accomplished by traversing A that is by accessing and Processing each element of A exactly once.
The following algorithm traverses a linear array. The simplicity of the algorithm comes from the fact that LA is a linear structure. Other linear structures such as linked list can also be easily traversed. On the other hand the traversal of non-linear structures such as trees and graphs is considerably more complicated.
\section{Task}  
\subsection{Procedure: Task 1 }     

\begin{figure*}
\centering
  \includegraphics[width=12cm,height=6cm,keepaspectratio]{123.png}
\caption{output}
\label{Figure:3}    
\end{figure*}
Write a C++ program to implement all the above described algorithms and display the following menu and ask the user for the desired operation.

\subsection{ }     

\begin{lstlisting}[language=Python]
 
#include<iostream>
using namespace std;
int main()
{
	int a[10],b=0,c;
	int d=0,co=0,n;
	loop:
	{
	cout<<"main menu"<<endl;
	cout<<"1: Enter element in the array \n";
	cout<<"2: for traverse the array \n";
	cin>>n;
	switch(n)
	{
		case'1':
			{
			cout<<"Enter the array less then 10: ";
			cin>>c;
			cout<<"Enetr the element in the array: "<<endl;
			while(b<c)
			{
				co++;
				cin>>a[b];
				b++;
			}
			}
		break;
		case'2':
			{
			cout<<"The traverse of array is: "<<endl;
			b=d;
			while(b<c)
			{
				cout<<a[b]<<" ";
				b++;
			}
			}
			break;
	}
}
		goto loop;
		return 0;
		
}
\end{lstlisting}

\section{Conclusion}  
In todays lab we have discussed how we  can create 1D and 2d array and how can we implement it on computer.

 
\end{document}                          % The required last line
