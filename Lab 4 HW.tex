\documentclass[11pt]{article}            % Report class in 11 points
\parindent0pt  \parskip10pt             % make block paragraphs
\usepackage{graphicx}
\usepackage{listings}
\graphicspath{ {images/} }
\usepackage{graphicx} %  graphics header file
\begin{document}
\begin{titlepage}
    \centering
  \vfill
    \includegraphics[width=8cm]{uni_logo.png} \\ 
	\vskip2cm
    {\bfseries\Large
	Data Structures and Algorithms \\ ( CS09203 )\\
	
	\vskip2cm
	Lab Report 
	 
	\vskip2cm
	}    


\begin{center}
\begin{tabular}{ l l  } 

Name: & M.Usman Ali \\ 
Registration \#: & SEU-F16-135 \\ 
Lab Report \#: & 04 \\ 
 Dated:& 30-04-2018\\ 
Submitted To:& Sir. Usman Ahmed\\ 

 %\hline
\end{tabular}
\end{center}
    \vfill
    The University of Lahore, Islamabad Campus\\
Department of Computer Science \& Information Technology
\end{titlepage}


    
    {\bfseries\Large
\centering
	Experiment \# 4 \\

Link list-Basic Insertion and transversal\\
	
	}    
 \vskip1cm
 \textbf {Objective}\\The objective of this session is to understand the various operations on linked list in C++.
using C++.. 
 
 \textbf {Software Tool}\\
1.  I use Code Blocks with GCC compiler.\\

\section{Theory }           LINKED LIST:-
A linked list is a collection of components, called nodes. Every node (except the last node)
contains the address of the next node. Thus, every node in a linked list has two components:
one to store the relevant information (that is, data) and one to store the address, called the link,
of the next node in the list. The address of the first node in the list is stored in a separate
location, called the head or first. Figure 1 is a pictorial representation of a node.

Figure 1

Linked list: A list of items, called nodes, in which the order of the nodes is determined by
the address, called the link, stored in each node.
The list in Figure 2 is an example of a linked
list.

Figure 2

The arrow in each node indicates that the address of the node to which it is pointing is stored
in that node. The down arrow in the last node indicates that this link field is NULL.
For a better understanding of this notation, suppose that the first node is at memory
location
\section{Task}  
\subsection{Procedure: Task 4 }     

\begin{figure*}
\centering
  \includegraphics[width=12cm,height=6cm,keepaspectratio]{lab401.png}
\caption{output}
\label{Figure:Untitled}    
\end{figure*}
Write a C++ code using functions for the following operations.
1.Creating a linked List.
2.Traversing a Linked List.

\subsection{ }     

\begin{lstlisting}[language=Python]
 
#include<iostream>
#include<stdlib.h>
using namespace std;
struct Node{
	int data;
	struct Node*next;
	
};
struct Node*head;
void insert(int x){
	struct Node*temp=(Node*)malloc(sizeof(struct Node));
	temp->data=x;
	temp->next=head;
	head=temp;
}
void print()
{
	struct Node*temp=head;
	cout<<"list is"<<endl;
	while(temp!=NULL)
	{
		cout<<temp->data;
		temp=temp->next;
		
	}
	cout<<endl;
}

int main(){
	head=NULL;
	cout<<"how many numbers"<<endl;
	int n,i,x,y;
	cin>>n;
	for(i=0;i<n;i++){
		cout<<"enter the number"<<endl;
		cin>>x;
		insert(x);
		print();	}	
}
\end{lstlisting}

\section{Conclusion}  
In today lab we have discussed how we can create a link list and display it on a screen.

 
\end{document}                          % The required last line

\end{lstlisting}

\section{Conclusion}  
In today lab we have discussed how we can create a link list and alose learn to delete a node and display it on a screen by having a code.

 
\end{document}                          % The required last line
